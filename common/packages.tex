%%%%%%%%%%%%%%%%%%%%%%%%%%%%%%%%%%%%%%%%%%%%%%%%%%%%
\usepackage[english]{babel}
\usepackage[utf8]{inputenc}
\usepackage[T1]{fontenc}
\usepackage{lmodern}

\usepackage{hyperref} % must be loaded before glossaries-extra

% bibliography
\usepackage[hyperref=true,backref=true,backend=biber,maxbibnames=9,maxcitenames=2,style=numeric,citestyle=numeric-comp,sorting=none,urldate=long]{biblatex} % hyperref uses links, backref goes back to citations, uses biber as backend, with 9 names at most in bibliography and 2 in citations, citing using numbers, and sorting in citation order
% sorting can be also ydnt for year descending, name, title or ynt for ascending year

\usepackage{adjustbox} % to resize boxes by keeping the same aspect ratio
\usepackage{algorithm} % algorithm environment
%\PassOptionsToPackage{noend}{algpseudocode}% comment out if want end's to show
\usepackage{algpseudocode} % improved pseudo-code
\usepackage{amsfonts}               %  AMS mathematical fonts
\usepackage{amsmath}
\usepackage{amssymb}                %  AMS mathematical symbols
\usepackage{amsthm}
\theoremstyle{definition}
\newtheorem{definition}{Definition}[section]
\theoremstyle{remark}
\newtheorem*{remark}{Remark}
\theoremstyle{definition}
\newtheorem{theorem}{Theorem}[section]
\usepackage{bm}                     %  black/bold mathematical symbols
\usepackage{booktabs}               %  better tables
\usepackage{calc}
\usepackage[labelfont=bf]{caption} % font=footnotesize % to have reduced caption font size
\usepackage{chronosys} % to draw timelines
\usepackage{csquotes}
\usepackage{enumitem} %left align the bulleted points
\usepackage{geometry}
\usepackage[%
    toc, % puts the link in the ToC record, % to use bib2gls
    abbreviations, % to load abbreviations / acronyms
    symbols, % to load symbols
    nonumberlist, % to avoid printing the numbers of the references in the acronyms page
    style=index, % in-line acronyms
]{glossaries-extra}
\usepackage{graphicx}               %  post-script images
%\usepackage{iwona} % extra fonts, substitute standard ones
\usepackage{listings} % to insert formatted code
\usepackage{lipsum} % for lorem ipsum text, not needed in the real work
\usepackage{makecell} % to change dimensions of cells, for math cases
\usepackage{mathtools} % for additional commands
\DeclarePairedDelimiter\ceil{\lceil}{\rceil}
\DeclarePairedDelimiter\floor{\lfloor}{\rfloor}
\usepackage{mfirstuc} % to have capitalization capabilities
\usepackage[final]{microtype}      % microtypography, final lets latex use it also in bibliography
\usepackage{multirow} % to allow for cells covering more than 1 row in tables
\newcommand{\STAB}[1]{\begin{tabular}{@{}c@{}}#1\end{tabular}}
\usepackage{nicefrac}       % compact symbols for 1/2, etc.
%\usepackage[lofdepth,lotdepth]{subfig}
\usepackage{ragged2e} % for justifying text
\usepackage{scrextend} % for multiple references to the same footnote
\usepackage{siunitx} % support for SI units of measurement and number typesetting
\usepackage[figuresright]{rotating}
%\usepackage{subfig}
\usepackage{subcaption} %  subfigure compatibility
\usepackage{svg} % for svg support, works only if inkscape is installed, default for Overleaf v2
\usepackage{tabularx} % equal-width columns in tables
\usepackage{textcomp} % extra fonts and symbols
\usepackage{url}            % simple URL typesetting
\usepackage{verbatim} % for extended verbatim support
\usepackage{xcolor} % to define colors and use standard CSS names add dvipsnames as option, but it clashes with xcolor loaded in toptesi, pay attention that if it goes in conflict with tikz/beamer, simply use \documentclass[usenames,dvipsnames]{beamer}, along with other custom options when defining the document class
\usepackage{xpatch}






% begin vertical rule patch for algorithmicx (http://tex.stackexchange.com/questions/144840/vertical-loop-block-lines-in-algorithmicx-with-noend-option)
\makeatletter
% start with some helper code
% This is the vertical rule that is inserted
\newcommand*{\algrule}[1][\algorithmicindent]{\makebox[#1][l]{\hspace*{.5em}\vrule height .75\baselineskip depth .25\baselineskip}}%

\newcount\ALG@printindent@tempcnta
\def\ALG@printindent{%
    \ifnum \theALG@nested>0% is there anything to print
        \ifx\ALG@text\ALG@x@notext% is this an end group without any text?
            % do nothing
            \addvspace{-3pt}% FUDGE for cases where no text is shown, to make the rules line up
        \else
            \unskip
            % draw a rule for each indent level
            \ALG@printindent@tempcnta=1
            \loop
                \algrule[\csname ALG@ind@\the\ALG@printindent@tempcnta\endcsname]%
                \advance \ALG@printindent@tempcnta 1
            \ifnum \ALG@printindent@tempcnta<\numexpr\theALG@nested+1\relax% can't do <=, so add one to RHS and use < instead
            \repeat
        \fi
    \fi
    }%
\usepackage{etoolbox}
% the following line injects our new indent handling code in place of the default spacing
\patchcmd{\ALG@doentity}{\noindent\hskip\ALG@tlm}{\ALG@printindent}{}{\errmessage{failed to patch}}
\makeatother
% end vertical rule patch for algorithmicx



\DeclareMathOperator*{\argmax}{arg\,max}
\DeclareMathOperator*{\argmin}{arg\,min}
\DeclareMathOperator*{\argsup}{arg\,sup}
\DeclareMathOperator*{\arginf}{arg\,inf}
\DeclarePairedDelimiterX{\dotp}[2]{\langle}{\rangle}{#1, #2}



\newcommand*{\inlineimg}[1]{%
    \raisebox{-.3\baselineskip}{%
        \includegraphics[
        height=\baselineskip,
        width=\baselineskip,
        keepaspectratio,
        ]{#1}%
    }%
}



% for not letting footnotes break in two pages
\interfootnotelinepenalty=1000000





%%%%%%%%%% NEW GLOSSARY STYLE
\newglossarystyle{myglosses}{
    \renewenvironment{theglossary}{
        \longtable{r l}
    }{
        \endlongtable
    }
    \renewcommand*{\glossaryheader}{}
    \renewcommand*{\glsgroupheading}[1]{}
    \renewcommand*{\glsgroupskip}{}
    \renewcommand*{\glsclearpage}{} 

     % set how each entry should appear:
      \renewcommand*{\glossentry}[2]{
        \glstarget{##1}{\textsc{\glossentryname{##1}}}
        &
        \glossentrydesc{##1} \\
    }
    \renewcommand*{\subglossentry}[3]{%
        \glossentry{##2}{##3}
    }
}