With the increasing awareness on environmental issues, research and development revolving around automotive industry – one of the most important yet pollutant industries world-wide – has gained momentum. In recent years, electric vehicles have been widely accepted as a clean and reliable alternative to fossil fuel vehicles, both in private and public transportation sectors, and are expected to quickly take over the market in the upcoming years.

Lithium-ion batteries have emerged as the main enabling technology in the development of EVs, mainly due to their high energy density and long lifespan. One of the key challenges posed by the spread of EV LIB packs is the real-time estimation of their state of health (SOH), commonly regarded as the main indicator of EV aging. However, SOH estimation is still a challenging task due to the electro-chemical complexity of LIBs and their non-linear charge and discharge dynamics.

In this thesis, the current solutions employed for SOH estimation of EV battery packs are discussed, and a novel real-time and computationally-inexpensive machine learning procedure for on-board SOH estimation is introduced, requiring just a narrow time window of voltage, current and state of charge measurements taken while driving the vehicle. To make up for the lack of large-scale publicly available EV monitoring data, a synthetic dataset has been generated by simulating multiple driving sessions of a Simulink-based EV model. The performances of our procedure are investigated and compared with those of other state-of-the-art methods in the field of time series extrinsic regression. Performances are evaluated both on a test set of synthetic data and on a real driving session monitoring dataset acquired from a private EV fleet management company.

Despite achieving promising results on synthetically-generated data, the procedure designed is still not able to perform well on real EVs, suggesting that further research is required. However, we claim that this procedure has much room for improvement with regards to its generalization capability, and we propose a path for future works to be developed.
